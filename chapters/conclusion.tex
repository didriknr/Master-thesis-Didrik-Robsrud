\chapter{Conclusions}\label{chap:Conclusions}
In conclusion, this thesis has demonstrated the increased safety of autonomous ground vehicle (AGV) navigation trough the fusion of radar - and lidar ranging data. The fusion of the ranging sensors allowed the system to detect objects that exists in the lidar's blind-zone, thereby enhancing the AVG's perception of its surroundings. Improved object avoidance was accomplished trough combining the strength of two different ranging sensor modalities.

The sensor fusion and navigation was implemented with the help of the ROS framework. The sensor drivers and the fusion was handled by ROS1. The ROS1 bridge was utilised to present the ranging data to the navigation system which runs on ROS2.

Testing was done to demonstrate the effects of the sensor fusion in a collision avoidance scenario, which displayed the effectiveness of the combined ranging sensors. Improved object detection and avoidance was demonstrated, and this lead to greater safety for AGV navigation. 

In summary, this thesis has proved the advantages introduced by the fusion of lidar and radar. The fusion of the ranging sensors, realised in ROS, presents a method for enhancing object detection and collision avoidance. 