\chapter{Discussions}\label{chap:Discussions}
This chapter will discuss some aspects of the project, like a failed prototype for the radar mount, assumptions about the radar FOV and why two radar modules was used. Some difficulties introduced by the Husky's skid steering will also be discussed. 

\section{Radar FOV}
It seems like a conservative assumption to say that the radar modules has a FOV of $\pm 28^\circ$ in azimuth, and  $\pm 14^\circ$ in elevation, compared to the antenna's bore sight. This is just the beam width of the antenna, given a gain drop of $3 dB$ and a radar frequency of $78 GHz$ \cite{xWR1843EvalModule}. The practical beam with is likely wider, but great improvements to the system are achieved even with this relatively narrow beam. %The antenna centre of both of the radar modules are positioned at about $333 mm$ above the ground.
Finding the "true" FOV of the radar modules antenna arrays consists of studying the antenna radiation patterns for a specific signal frequency and a specific gain.

\section{Failed radar mount}
The mounting system has been trough a failed prototype before taking its current form. The four pillars in the middle of the first prototype of the radar mount where wider than in the current mount. Those pillars conflicted with the pin mentioned in section \ref{subsec:Radar}, this was attempted fixed by filing down the PLA, which worked to some degree. The walls of the mentioned pillars where so far apart that the nut used could freely rotate, which created the need for a pinching tool when mounting the radar modules. The two pillars sitting on the side clamping structures where thinner than the other pillars, too thin in fact. These pillars would not fit the nut between their walls. The issue was attempted solved by cutting the walls of one of the pillars with pliers, which ended up breaking the entire pillar off. Some adjustment was made to the design, since it had to be re-printed anyway. Among these changes was turning the edges where the bigger clamping structure meets the flat area to fillet-like edges. This change was also transferred to the generic mount, but the new generic mount was newer printed, as the first version still functioned as it should, and the change had little to no practical impact.

\section{Radar amount and placement}
Two radar modules are used instead of one, and they are positioned similarly, so why use two? The main reason two radar modules is used is to increase the density of the points produced by the radars collectively. This might be useful for future machine vision projects. A different reason was to see if more radar modules could be used together. The radar mount sits on top of the front bumper, but it should be possible to fasten it underneath. This may further improve the systems capability to detect objects that are close and low to the ground, as the antenna centres would end up much lower. However, this might make them more susceptible to read the ground as a object, and it might leave the modules in a more vulnerable position if a coalition where to occur during testing. The radar modules could have been angled differently to increase the FOV at the cost of decreased point density. This could have been realised by angling the radar modules so they "look" more out to the sides by re-designing the radar mount. A new radar mount could also have been printed and mounted to the back bumper, then a radar module could have observed the front of the Husky, and one observed the back. This is likely useful for certain applications.

\section{Skid steering}
The Husky steers trough skidding, which means that the right wheels turn a at a different angular velocity than the left wheels. This is similar to how some belt based vehicles turns, like excavator. It is difficult to calculate how such a vehicle will turn based on its wheel/belt movement, due to the turning being highly reliant on the friction between the ground surface and the wheels/belts. This results in poor wheel odometry being produced by this turning mechanism being highly unreliable, thus making accurate localisation more difficult. This ended up representing an annoyance during the project. The system would greatly benefit form better localisation, ether trough hardware or software, or both.